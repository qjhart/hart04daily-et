\documentclass[letterpaper]{article}

\usepackage{qjh-article}
\renewcommand{\familydefault}{cmss}

\newcommand{\hel}{Heliosat}

\title{Daily Evapotranspiration - Radiation Model} 
\author{Quinn Hart qjhart@ucdavis.edu}

\date{Apr 2003}

\abstract{  
  This document gives a brief description of the solar radiation
  model, the choice of model, and some justification of the model.  We
  have begun with a model that is a combination of model
  \hel-I~\cite{beyer.ea:96:modif-of} and
  \hel-II~\cite{rigol.ea:00:on-clear,lefer.ea:02:descr-of} This paper
  is a comparison of the differences between these models and the
  2001~\cite{gauti.ea:97:surfa-solar,froui.ea:89:simpl-analy} model.
}
\begin{document}

\maketitle

\section{Solar Radiation Model}

The \hel-I model used a simple solar irradiance prediction model
coupled with cloudiness parameters based on ratios of visible pixels
from the Satellite.  The \hel-II model updates the solar radiation
model and also determines surface albedos by modeling the diffuse
radiance, and transmission of the atmosphere.  The 2001 model is a
simple model that describes a one layer atmospheric model.  Ground and
cloud albedos are calculated from this model and then used to predict
the surface insolation.

We have chosen to implement the radiative transfer from the \hel-II
model, while retaining the ratio based approach of albedos to
determine the cloudiness factors and their contribution to the solar
insolation.

In the future, we plan to include both the \hel-II albedo estimations,
as well as compare insolation predictions from the 2001 model.  The
following describes some of the main differences between the \hel based
models and the 2001 model.

\begin{description}

\item[Basic Radiative Model] 
  
  The Heliosat model basically combines aspects of aerosols, relative
  humidity, ozone, and molecular scattering into a single parameter,
  the Linke turbidity, $T_l$, which relates the optical depth for an
  arbitrary atmosphere to one of a Rayleigh only scattering
  atmosphere.  Then, besides geometry, elevation, and some second
  order effects, the predicted insolation is a simple calculation
  using this parameter.  For example, the direct beam calculation is
  $G_direct ~ e^{-0.8662t_l(p/p_0)\gamma_r}$ where $T_l$ is the Linke
  turbidity factor, $p/p_0$ is a elevation correction, and $\gamma_R$
  is the rayleigh optical depth.  We have seasonal estimations of the
  Linke turbidity from a world database of this information.
  
  The 2001 model on the other hand, uses six parameters to estimate
  the atmospheric effects.
  
  The \hel model's simplicity allows use an easier formulation for
  modeling predicted changes in radiation.  For example, we can use
  the seasonal model of Linke turbidity, but potentially make spatial
  changes to this estimation by comparing to the CIMIS radiation
  measurements.
  
  One potential problem with using a model based on the Linke
  Turbidity is that it could conceivably be more difficult to tie the
  Linke Turbidity to the CIMIS based relative humidity measurements,
  which is another possible way to estimate the turbidity.
  
\item[Integration] 
 
  The \hel-II model uses an analytical integration over solar angles,
  as opposed to other models which use approximations over
  instantaneous values.  This has a number of advantages.  The largest
  is in filling gaps caused by missing datasets.  If for example, we
  have GOES satellite measurements from 9,11 am, we can simply fill in
  the missing 10 am value by extending the range of the other
  measurements.  Using an analytical integration will assign the
  proper weights to the bounding measurements, in this case assigning
  a greater weight to the measurement at 11.
  
  Even if we decide to use a different radiative transfer model, we
  can still take advantage of the integration methods developed here,
  although the integration functions are related directly to
  formulation of the radiation as described in the heliosat model.


\item[Diffuse Radiation]
  
  The 2001 model uses a simple model that basically combines direct
  and diffuse radiance


\item[Cloud Cover] 
  
  One of the major differences between the Heliosat based model, and
  the 2001 model is in the calculation of the effects of cloud cover.
  The heliosat model uses an empirical relation that roughly linearly
  relates a measured cloud brightness with the clear-sky fraction.
  This empirical relation has been shown to be valid in a number of
  studies~\cite{beyer.ea:96:modif-of}. The 2001 model instead uses a
  simple cloud model formulation to determine cloud albedo from pixel
  brightness.  This results in a quadratic functional relationship
  between cloud brightness and downwelling radiance.  Since the clear
  sky \hel2 model does not include cloud cover, we could compare these
  models at a later date.


\item[Albedos] 
  
  One potential advantage of the 2001 model is that the model predicts
  absolute albedo values, whereas the \hel1 model uses relative
  brightness comparisons.  We would like to compare the \hel1 and
  \hel2 model formulations, and at that point, we could calculate
  absolute albedos with

\end{description}

\subsection{Comparisons}

If we would like to compare the \hel model with the 2001 model, we
would follow the following steps.  

\begin{itemize}
  
\item Calculate true surface albedos using scattering and absorption
  cooeficients based either on the \hel2 method or 2001 method. 
  
\item Compute cloud albedo using the 2001 model.  As discussed above,
  this is a quadratic equation that usually has a simple empirical
  formula for the cloud absorption.
  
\item Using the computed cloud albedo, and the 2001 cloud model,
  compute the 2001 predicted surface radiance.  
  
\item Compare the two predicted radiance values to determine where the
  models differ, and using the CIMIS stations as an estimator of
  accuracy.

\end{itemize}


\bibliographystyle{alpha}
\bibliography{cimis}

\end{document}
